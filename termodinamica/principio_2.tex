\section{Secondo principio della termodinamica}

È impossibile realizzare un processo che abbia come unico risultato la trasformazione in lavoro del calore fornito da una sorgente a temperatura uniforme
\begin{equation}
	\eta = 1 - \frac{T_1}{T_2}
\end{equation}
\begin{equation}
	L_\mathrm{max} = Q_a \eta_R = Q_a \left( 1 - \frac{T_1}{t_2} \right)
\end{equation}



\subsection{Teorema di Clausius}

Per una macchina che H compie un ciclo reversibile si verifica: $\oint \frac{dQ}{T} =  0$.
Se la macchina compie un ciclo irreversibile, il risultato dell’integrale è < 0.



\subsection{Entropia}

Il valore dell’integrale $\int_A^B \left( \frac{dQ}{T} \right)_\mathrm{rev} $, esteso ad una qualunque trasformazione reversibile che congiunge due stati di un sistema termodinamico, è sempre lo stesso, cioè non dipende dalla particolare trasformazione reversibile scelta per il calcolo.