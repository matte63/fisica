\section{Trasformazioni e lavoro dei gas}


\subsection{Lavoro di un gas}

\subsubsection*{Relazione di Mayer}
\begin{equation}
	c_p - c_v = R
\end{equation}

\subsubsection*{Entalpia}
\begin{equation}
	H = U + pV
\end{equation}
\begin{equation}
	\Delta H = n \int_{T_a}^{T_b} c_p \ dT
\end{equation}



\subsection{Trasformazioni}

\subsubsection*{Equazione di stato dei gas perfetti}
\begin{equation}
	\highlight{pV = nRT}
\end{equation}

\subsubsection*{Equazione di stato dei gas reali}
\begin{equation}
	pV = nRT
\end{equation}
\begin{equation*}
	R = p_0 V_m \alpha = \SI{8.314}{\joule \per \mol \kelvin}
\end{equation*}



\subsubsection{Trasformazioni isobare}

\begin{equation*}
	\frac{V}{T} = \frac{nR}{p}
\end{equation*}
da cui
\begin{equation}
	\frac{V_1}{T_1} = \frac{V_2}{T_2}
\end{equation}

Il lavoro è calcolato come
\begin{equation*}
	\Delta U = n c_V \Delta T
\end{equation*}
da cui
\begin{equation*}
	L = p (V_B - V_A) = p \left(\frac{n R T_B}{p} - \frac{n R T_A}{p}\right)
\end{equation*}

\begin{equation}
	\highlight{L = n R (T_B - T_A)}
\end{equation}

Mentre il calore
\begin{equation}
	Q = n c_p (T_B - T_A)
\end{equation}
grazie alla relazione dei Mayer $c_p = c_V + R$



\subsubsection{Trasformazioni isoterma}



\subsubsection{Trasformazioni isocora}



\subsubsection{Trasformazioni adiabatiche}



