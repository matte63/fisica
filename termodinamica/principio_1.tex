\section{Primo principio della termodinamica}



\subsection{Equilibrio termodinamico}

Un sistema si trova in \emph{equilibrio termodinamico} quando:
\begin{enumerate}[noitemsep]
	\item \emph{le forze meccaniche} che si esercitano sulle varie parti del sistema \emph{sono in equilibrio}
	\item \emph{non v'è moto macroscopico} osservabile fra le varie parti
	\item \emph{tutte le parti} del sistema \emph{sono alla medesima temperatura}
	\item eventuali \emph{reazioni chimiche} \emph{hanno raggiunto l'equilibrio}
	\item i processi di \emph{cambiamenti di stato} \emph{hanno raggiunto l'equilibrio}
\end{enumerate}



\subsection{Lavoro in trasformazioni reversibili}

\begin{equation}
	\Delta L = \pm F \Delta l = p \Delta V
\end{equation}

\begin{equation}
	L_{AB} = \int_A^B p \ dV
\end{equation}



\subsection{Calore e lavoro}

\begin{equation}
	L = J Q
\end{equation}
con $J = \SI{4.1868}{\joule\per\cal}$



\subsection{Conservazione dell'energia}


Se un sistema compie una trasformazione dallo stato A allo stato B, scambiando calore e lavoro con l’ambiente, $Q$ e $L$ dipendono dalla trasformazione che congiunge i due stati termodinamici, mentre la differenza $Q - L$ risulta indipendente dalla trasformazione.

\begin{equation}
	U_2 - U_1 = Q - L
\end{equation}
\begin{equation}
	dU = dQ - dL
\end{equation}
o più esplicitamente
\begin{equation}
	dU = dQ - p \ dV + F \ dl + H \ dM + \mu \ dn + \dots
\end{equation}

