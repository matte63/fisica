\section{Termologia}



\subsection{Scale termometriche}

Si prendano in considerazione volume e pressione di una data massa di un gas perfetto, con la temperatura $t$ espressa in gradi centigradi.

A pressione costante il volume sarà
\begin{equation}
	\highlight{
		V = V_0 (1 + \alpha t)
	}
\end{equation}

A volume costante la pressione sarà
\begin{equation}
	\highlight{
		p = p_0 (1 + \alpha t)
	}
\end{equation}

Con $p_0$ e $V_0$ la pressione e il volume a 0°C e $\alpha = 1/273.15$ il coefficiente uguale per tutti i gas perfetti. La temperatura assoluta $T$ sarà legata alla temperatura centigrada $t$ dalla relazione $T = 273.15 + t$.

Quindi più semplicemente
\begin{align*}
	V & = \frac{V_0}{273.15}T
	  &
	p & = \frac{p_0}{273.15}T
\end{align*}



\subsection{Espansione termica dei solidi}

\begin{equation*}
	l = l(t)
\end{equation*}

\begin{equation}
	\highlight{
	l = l_0 [ 1 + \alpha (t) \Delta t ]
	}
\end{equation}
con $\alpha (t)$ coefficiente di espansione lineare definito da
\begin{equation}
	\highlight{
		\alpha (t) = \frac{1}{l_0}\frac{dl}{dt}
	}
\end{equation}

\subsection{Quantità di calore e calore specifico}

Il calore specifico rappresenta il calore che occorre scambiare con l’unità di massa di una data sostanza alla temperatura $T$, per far variare la temperatura di \SI{1}{\kelvin}.
\begin{equation*}
	dQ = \mathrm{mc} \ dT
\end{equation*}
\begin{equation}
	\highlight{
		Q = c m (t_2 - t_1)
	}
\end{equation}