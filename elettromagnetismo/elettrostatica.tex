\section{Elettrostatica}



\subsection{Corrente elettrica}

Si \textcolor{accent}{definisce intensità di corrente} il rapporto tra la carica $dq$ che passa per una sezione di filo in un intervallo di tempo $dt$

\begin{equation}
    i = \frac{dq}{dt}
\end{equation}



\subsection{Legge di Coulomb}

\begin{figure}[!h]
    \centering
    \begin{tikzpicture}
        \coordinate (q1) at (0,0);
        \coordinate (q2) at (3,.3);

        \draw[black, thick, -latex] (q1) -- (q2) node[pos=.5, anchor=south] {$\vect{r}$};

        \filldraw[light] (q1) circle (.1);
        \draw[black, thin] (q1) circle (.1) node[anchor=north west] {$q_1$};

        \filldraw[light] (q2) circle (.1);
        \draw[black, thin] (q2) circle (.1) node[anchor=north west] {$q_2$};
    \end{tikzpicture}
    \caption{}
\end{figure}

\begin{equation}
    \vect{F} = k_0 \frac{q_1 q_2}{r^2} \frac{\vect{r}}{r}
\end{equation}

con $k_0 = \SI{8.98776e9}{\newton \meter^2 \per \coulomb^2}$.

Il \textcolor{accent}{coulomb} è la carica che passa in un secondo attraverso la sezione di un conduttore filiforme percorso da un a corrente costante e di intensità pari a $\SI{1}{\ampere}$.

\begin{equation}
    q = i \ t
\end{equation}

Per semplificare alcuni calcoli viene posta $k_0 = \frac{1}{4 \pi \varepsilon_0}$  e si introduce la \emph{costante dielettrica del vuoto}

\begin{equation}
    \varepsilon_0 = \frac{1}{4 \pi k_0} = \SI{8,85415e-12}{\coulomb^2 \per \newton \meter^2}
\end{equation}


La \textcolor{accent}{legge di Coulomb} di conseguenza è

\begin{equation}
    \highlight{
        \vect{F} = \frac{1}{4 \pi \varepsilon_0} \frac{q_1 q_2}{r^2} \frac{\vect{r}}{r}
    }
\end{equation}



\subsection{Quantizzazione della carica}

\begin{equation}
    e = \SI{1.60206e-19}{\coulomb}
\end{equation}



\subsection{Energia elettrostatica}

\begin{equation}
    F(r) = - \frac{\partial \mathcal{U}}{\partial r} = \frac{1}{4 \pi \varepsilon_0} \frac{q_1 q_2}{r^2}
\end{equation}

\begin{equation}
    \vect{F} = - \nabla \mathcal{U}
\end{equation}

L'energia potenziale di un sistema di cariche puntiformi è definita come

\begin{equation}
    \highlight{
    \mathcal{U} = \frac{1}{2} \frac{1}{4 \pi \varepsilon_0} {\sum}_j {\sum}_{i \ne j} \frac{q_i q_j}{r_{ij}}
    }
\end{equation}



\subsection{Campo elettrico}

\begin{equation}
    \highlight{
        \vect{E}_0 = \lim_{q \to 0} \frac{\vect{F}}{q}
    }
\end{equation}

\begin{figure}[!h]
    \centering
    \begin{tikzpicture}
        \coordinate (O) at (0,0);
        \coordinate (q) at (1,.5);
        \coordinate (P) at (2,2.5);
        \draw[black, thin, -latex] (O) -- (-1,-1) node[anchor=north west] {$x$};
        \draw[black, thin, -latex] (O) -- (5,0)   node[anchor=north]      {$y$};
        \draw[black, thin, -latex] (O) -- (0,3)   node[anchor=north east] {$z$};

        \draw[black, thick, -latex] (q) -- (P)
        node[anchor=north west] {$P$}
        node[pos=.5, anchor=north west] {$\vect{r}$};
        \filldraw[black] (P) circle (1pt);
        \filldraw[light] (q) circle (.1);
        \draw[black, ultra thin] (q) circle (.1)
        node[anchor=north west] {$q$};


    \end{tikzpicture}
    \caption{}
\end{figure}

Il campo elettrico creato da una carica puntiforme $q$ in un punto ($x$, $y$, $z$) individuato da $\vect{r}$ ha intensità

\begin{equation}
    \highlight{
        \vect{E}_0 = \frac{1}{4 \pi \varepsilon_0} \frac{q}{r^2} \frac{\vect{r}}{r}
    }
\end{equation}

Inoltre il campo creato da $N$ cariche puntiformi ha intensità

\begin{equation}
    \vect{E}_0 = \frac{1}{4 \pi \varepsilon_0} {\sum}_i \frac{q_i}{r_i^2} \frac{\vect{r}_i}{r_i}
\end{equation}

In una regione di spazio in cui si ha una \emph{distribuzione continua di carica} è possibile definire una \textcolor{accent}{densità spaziale di carica}

\begin{equation}
    \highlight{
        \rho (x,y,z) = \frac{d q}{d \tau}
    }
\end{equation}

Di conseguenza il campo elettrico sarà

\begin{equation}
    \highlight{
        \vect{E}_0 (x,y,z) = \frac{1}{4 \pi \varepsilon_0} \int_\tau \frac{\rho (x,y,x)}{r^2} \frac{\vect{r}}{r} d \tau
    }
\end{equation}

Nel caso in cui le cariche siano distribuite lungo una superficie $S$ è possibile definire la \textcolor{accent}{densità superficiale di carica}

\begin{equation}
    \highlight{
        \sigma(x,y,z) = \frac{dq}{dS}
    }
\end{equation}


Di conseguenza il campo elettrico sarà

\begin{equation}
    \highlight{
        \vect{E}_0 (x,y,z) = \frac{1}{4 \pi \varepsilon_0} \int_S \frac{\sigma (x,y,x)}{r^2} \frac{\vect{r}}{r} d S
    }
\end{equation}



\subsection{Flusso di un vettore}

Il \textcolor{accent}{flusso di un vettore} è l'elemento definito come

\begin{equation}
    \highlight{
        \phi (\vect{E}_0) = \int_S \vect{E_0} \cdot \vect{n} \ dS
    }
\end{equation}



\subsection{Legge di Gauss}

Il teorema della divergenza di Gauss definisce

\begin{equation}
    \highlight{
        \int_S \vect{E} \cdot \vect{n} \ dS = \int_\tau \nabla \cdot \vect{E} \ d \tau
    }
\end{equation}

Inoltre per una distribuzione di cariche puntiformi è vero che

\begin{equation}
    \phi_S (\vect{E}_0) = \int_S \vect{E}_0 \cdot \vect{n} \ dS = \frac{\sum q}{\varepsilon_0}
\end{equation}

In generale il \textcolor{accent}{teorema di Gauss} viene definito come

\begin{equation}
    \label{eq:teorema_generale_Gauss}
    \highlight{
        \phi_S (\vect{E}_0) = \int_S \vect{E}_0 \cdot \ dS =
        \frac{1}{\varepsilon_0} \int_\tau \rho (x,y,z) \ d\tau
    }
\end{equation}