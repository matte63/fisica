\chapter{Operazioni}



\section{Gradiente}

Il \textcolor{accent}{gradiente} di una funzione scalare $A$ è un vettore che ha come componenti nelle direzioni $x$, $y$ e $z$ le derivate parziali rispettivamente in $\partial x$, $\partial y$ e $\partial z$
\begin{equation}
    \mathrm{grad} A = \nabla A = \frac{\partial A}{\partial x} \vect{i} +
    \frac{\partial A}{\partial y} \vect{j} +
    \frac{\partial A}{\partial z} \vect{k}
\end{equation}



\section{Divergenza}

La \textcolor{accent}{divergenza} è una grandezza scalare definita da
\begin{equation}
    \mathrm{div} \vect{A} =
    \nabla \cdot \vect{A} = \frac{\partial A_x}{\partial x} \vect{i} +
    \frac{\partial A_y}{\partial y} \vect{j} +
    \frac{\partial A_z}{\partial z} \vect{k}
\end{equation}
e $\nabla$ rappresenta l'operatore
\begin{equation}
    \highlight{
        \vect{\nabla} = \frac{\partial}{\partial x} \vect{i} +
        \frac{\partial}{\partial y} \vect{j} +
        \frac{\partial}{\partial z} \vect{k}
    }
\end{equation}



\section{Rotore}

Il \textcolor{accent}{rotore} è invece definito da
\begin{equation}\begin{split}
        \highlight{
            \mathrm{rot} \vect{v}
        }
        & = \vect{i} \left( \frac{\partial v_z}{\partial y} - \frac{\partial v_y}{\partial z} \right) +
        \vect{j} \left( \frac{\partial v_x}{\partial z} - \frac{\partial v_z}{\partial x} \right) +
        \vect{k} \left( \frac{\partial v_y}{\partial x} - \frac{\partial v_x}{\partial y} \right) = \\
        & =
        \highlight{
            \begin{vmatrix}
                \vect{i}                    & \vect{j}                    & \vect{k}                    \\
                \frac{\partial}{\partial x} & \frac{\partial}{\partial y} & \frac{\partial}{\partial z} \\
                v_x                         & v_y                         & v_z
            \end{vmatrix}
        }
    \end{split}
\end{equation}

Notazioni equivalenti sono
\begin{equation}
    \mathrm{rot} \vect{v} = \mathrm{curl} \vect{v} = \nabla \times \vect{v}
\end{equation}