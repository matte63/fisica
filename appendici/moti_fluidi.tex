\chapter{Moti rotazionali e irrotazionali}

In un sistema continuo, dato un elemento infinitesimo nell'intorno di un generico punto $O$ si definiranno come $d\vect{P}$ gli spostamenti elementari degli elementi nell'intorno si $O$. Questi spostamenti avranno tre componenti:
\begin{enumerate}
	\item[$d\vect{P}_1$] che rappresenta una traslazione;
	\item[$d\vect{P}_2$] che rappresenta una rotazione intorno a $O$;
	\item[$d\vect{P}_3$] legata alla deformazione dell'elemento.
\end{enumerate}

I moti definiti \textcolor{accent}{irrotazionali} sono quelli per i quali le rotazioni elementari delle particelle sono identicamente nulle. Nei moti irrotazionali le particelle non hanno momento della quantità di moto e hanno espressioni della velocità relativamente semplici.

Se invece il sistema continuo presenta punti per i quali gli elementi nel loro intorno hanno una velocità angolare risultante intorno a quei punti, il moto si dice \textcolor{accent}{rotazionale} o vorticoso.

Per descrivere i moti rotazionali è utile definire il vettore \textcolor{accent}{rotore della velocità}:
\begin{equation}\begin{split}
		\highlight{
			\mathrm{rot} \vect{v}
		}
		& = \vect{i} \left( \frac{\partial v_z}{\partial y} - \frac{\partial v_y}{\partial z} \right) +
		\vect{j} \left( \frac{\partial v_x}{\partial z} - \frac{\partial v_z}{\partial x} \right) +
		\vect{k} \left( \frac{\partial v_y}{\partial x} - \frac{\partial v_x}{\partial y} \right) = \\
		& =
		\highlight{
			\begin{vmatrix}
				\vect{i}                    & \vect{j}                    & \vect{k}                    \\
				\frac{\partial}{\partial x} & \frac{\partial}{\partial y} & \frac{\partial}{\partial z} \\
				v_x                         & v_y                         & v_z
			\end{vmatrix}
		}
	\end{split}
\end{equation}

Se il vettore $\mathrm{rot} \vect{v}$ è 0 ovunque il moto è irrotazionale, se è 0 intorno a un punto l'elemento nell'intorno del punto non ha una velocità angolare risultante e il moto è irrotazionale nell'intorno considerato.
Il \emph{significato fisico} è illustrato dal fatto che se un rotore è non nullo in un punto, introdotto un sistema di coordinate in moto rotatorio con velocità $\vect{\omega} = \frac{1}{2} \mathrm{rot} \vect{v}$, in tale sistema il moto nell'intorno di $P$ è irrotazionale e $\frac{1}{2} \mathrm{rot} \vect{v}$ la velocità angolare del sistema continuo intorno a quel punto.

Inoltre se il campo del vettore $\vect{v}$ è irrotazionale, vale per qualsiasi linea chiusa la relazione:
\begin{equation}
	\highlight{
		\oint \vect{v} \cdot d\vect{s} = \oint v_s ds = 0
	}
	\label{eq:oint_vec_nullo}
\end{equation}
con $ds$ elemento della curva. Per forze conservative vale analogamente
\begin{equation}
	\oint \vect{F} \cdot d\vect{s} = 0
\end{equation}
che permette l'introduzione della funzione energia potenziale. Per i campi irrotazionali si può quindi definire una funzione $\psi$ della velocità:
\begin{equation}
	\begin{dcases}
		v_x = - \frac{\partial \psi}{\partial x} \\
		v_y = - \frac{\partial \psi}{\partial y} \\
		v_z = - \frac{\partial \psi}{\partial z}
	\end{dcases}
\end{equation}

Se la proprietà (\ref{eq:oint_vec_nullo}) non è rispettata anche solo lungo una linea nel campo, allora questo non è irrotazionale ma è possibile che esistano delle regioni limitate in cui rimane valida e in tali regioni è possibile definire la funzione $\psi$.