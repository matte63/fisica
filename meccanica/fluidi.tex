\section{Meccanica dei fluidi}

I \textcolor{accent}{fluidi ideali} godono delle seguenti proprietà

\begin{enumerate}
    \item incompressibilità
    \item assenza di viscosità, quindi i cambiamenti di forma non avvengono a spese del lavoro
\end{enumerate}



\subsection{Pressione in un punto di un fluido}

La \textcolor{accent}{pressione media} esercitata dalla componente normale di $\Delta \vect{F}$ per l'elemento $\Delta S$ è rappresentata con

\begin{equation*}
    p_m = \frac{\Delta \vect{F} \cdot \vect{n}}{\Delta S}
\end{equation*}

e la \textcolor{accent}{pressione} come il limite

\begin{equation}
    p = \lim_{\Delta S \to 0} \frac{\Delta \vect{F} \cdot \vect{n}}{\Delta S}
\end{equation}

Nel caso di fluidi ideali o in equilibrio (e $\Delta \vect{F}$ diretto come $\vect{n}$) è definita come

\begin{equation}
    \highlight{
        p = \lim_{\Delta S \to 0} \frac{|\Delta \vect{F}|}{\Delta S}
    }
\end{equation}



\subsection{Equazioni della statica dei fluidi}

Considerato un elemento $dV$ del fluido in equilibrio

\begin{equation}
    \vect{g} = \frac{d \vect{F}_v}{dm}
\end{equation}

Le \textcolor{accent}{equazioni della statica dei fluidi} sono

\begin{equation}
    \highlight{
        \nabla p = \rho \vect{g} =
        \begin{cases}
            \dfrac{\partial p}{\partial x} = \rho X \\
            \dfrac{\partial p}{\partial y} = \rho Y \\
            \dfrac{\partial p}{\partial z} = \rho Z
        \end{cases}
    }
\end{equation}

\begin{equation*}
    dp = - \rho \ dU
\end{equation*}

In un fluido omogeneo ($\rho = \mathrm{const}$) si ha

\begin{equation}
    \highlight{
        p = - \rho U + \mathrm{const}
    }
\end{equation}

\begin{quote}
    In un fluido omogeneo soggetto a forze di volume conservative e in equilibrio le \textcolor{accent}{superfici isobare} ($p = \mathrm{const}$) sono superfici equipotenziali ($U = \mathrm{const}$).
\end{quote}



\subsection{Meccanica dei fluidi pesanti}



\subsection{Principio di Pascal}

\begin{quote}
    In un fluido in quiete non soggetto a forze esterne di volume la pressione è uguale in tutto il fluido.
\end{quote}



\subsection{Misura della pressione}

\begin{equation}
    p = p_0 + \rho g h
\end{equation}



\subsection{Principio di Archimede}

\begin{figure}[!h]
    \centering
    \begin{tikzpicture}
        \draw[black, thin] (-2,0) -- (4,0);
        \filldraw[gray, pattern=dots, pattern color=lightgray] (0,0) rectangle ++ (2,1.5);
        \draw[black, ultra thick] (0,1.8) -- (0,0) -- (2,0) -- (2,1.8);
        \filldraw[gray, pattern=crosshatch, pattern color=lightgray] (.5,1) rectangle ++ (1,.5);
        \filldraw[gray, pattern=north east lines, pattern color=lightgray] (.5,1.5) rectangle ++ (1,.5);
        \draw[gray, thin] (.5,1) rectangle ++ (1,1);
        \draw[black, -latex, thick] (.9,1.25) -- ++(0,.67) node[anchor= east, pos=.5] {$\vec{S}$};
        \draw[black, -latex, thick] (1.1,1.5) -- ++(0,-1) node[anchor= west, pos=.5] {$\vec{P}$};

        % \draw[black, ultra thin, step=.5] (-2,0) grid (4,3);
        % \draw[black, very thin, step=1] (-2,0) grid (4,3);
    \end{tikzpicture}
    \caption{}
    \label{fig:principio_archimede}
\end{figure}

\begin{quote}
    Un corpo immerso in un fluido pesante in equilibrio, è soggetto a un forza diretta verticalmente verso l'alto di intensità pari al peso della massa fluida spostata e applicata al baricentro di questa.
\end{quote}



\subsection{Dinamica dei fluidi}

Nello studio delle particelle che compongono un fluido si possono seguire due diversi procedimenti. Il primo, proposto da Eulero conduce alle equazioni di Lagrange ed è detto \textcolor{accent}{lagrangiano}. Studia il moto delle singole particelle e delle loro coordinate $x$, $y$, $z$ come funzione di tempo:
\begin{align*}
    x & = x(x_0, y_0, z_0, t) & y & = y(x_0, y_0, z_0, t) & z & = z(x_0, y_0, z_0, t)
\end{align*}
con $x_0$, $y_0$, $z_0$ le coordinate della particella nell'istante $t = t_0$.

Il secondo procedimento, sempre proposto da Eulero, invece di seguire le singole particelle ne studia densità e velocità in funzione del tempo, nello spazio occupato dal fluido. Questo punto di vista è detto \textcolor{accent}{euleriano} ed è definito dalle seguenti equazioni:
\begin{align*}
    \rho & = \rho (x,y,z,t) & \vect{v} & = \vect{v} (x,y,z,t)
\end{align*}

Il moto di un fluido si dice \textcolor{accent}{stazionario} quando la velocità \vect{v} di ogni punto dello spazio non cambia con il tempo. Ovviamente la velocità può variare da punto a punto e la velocità per un punto rimane uguale per tutte le particelle che passano per quel punto.

Si definisce \textcolor{accent}{non stazionario} se la velocità in un punto cambia con il tempo.
Questa può verificarsi nei transienti quando variano le condizioni del moto.
Viene definito tra i moti \textcolor{accent}{non rotazionale} il moto \textcolor{accent}{turbolento} caratterizzato da variazioni rapidi e casuali della velocità da punto a punto.




\subsection{Linee di flusso e di corrente}



\subsection{Equazione di continuità}

\begin{equation}
    \highlight{
        \rho_1 v_1 S_1 = \rho_2 v_2 S_2
    }
\end{equation}



\subsection{Equazione di Bernoulli}

\begin{equation}
    \highlight{
        z + \frac{p}{\rho g} + \frac{v^2}{2g} = \mathrm{const}
    }
\end{equation}



