\section{Meccanica dei sistemi di punti materiali}



\subsection{Moto del centro di massa}

È possibile studiare il moto di un qualunque sistema materiale riducendo lo studio ad un numero sufficientemente elevato di punti materiali in modo da avere $n$ vettori spostamento $\vect{r}_{j = 1, \dots , n}$ grazie ai quali è possibile determinare ogni singola forza $\vect{F}_j$ agente su ciascun punto materiale tale che

\begin{equation}
    \vect{F}_j = m_j \frac{d^2 \vect{r}_j}{d t^2}
\end{equation}

Per ciascuna di queste forze è necessario individuare il contributo \textcolor{accent}{interno} $\vect{F}_j^{\mathrm{(int)}}$ dato da punti appartenenti al sistema e quello \textcolor{accent}{esterno} $\vect{F}_j^{\mathrm{(est)}}$ dato da punti esterni al sistema.

\begin{equation}
    \vect{F}_j^{\mathrm{(est)}} + \vect{F}_{jk}^{\mathrm{(int)}} = m_j \frac{d^2 \vect{r}_j}{d t^2}
\end{equation}

posta $\vect{F}_j^{\mathrm{(int)}} = \sum_{k=j} \vect{F}_{jk}^{\mathrm{(int)}}$, con $\vect{F}_{jk}^{\mathrm{(int)}}$ la forza agente sul punto $j$ da parte di un altro punto $k$

Sommando le $n$ equazioni definite sarà verificato che per ogni forza $\vect{F}_{jk}^{\mathrm{(int)}}$ ve ne sarà una $\vect{F}_{kj}^{\mathrm{(int)}}$ di reazione.

Si otterrà quindi che

\begin{equation}
    \vect{F}_j^{\mathrm{(est)}} = \vect{F} = {\sum}_j m_j \frac{d^2 \vect{r}_j}{d t^2}
\end{equation}

Posta la massa $m = {\sum}_j m_j$ è possibile individuare il \textcolor{accent}{centro di massa} $C$ attraverso il vettore

\begin{equation}
    \highlight{
        \vect{r}_c = \frac{\sum m_j \vect{r}_j}{m}
    }
\end{equation}

di componenti

\begin{equation*}
    \begin{aligned}
        x_c &= \frac{1}{m} {\sum}_j m_j x_j \\
        y_c &= \frac{1}{m} {\sum}_j m_j y_j \\
        z_c &= \frac{1}{m} {\sum}_j m_j z_j
    \end{aligned}
\end{equation*}

È quindi possibile scrivere

\begin{equation}
    \highlight{
        \vect{F} = m \frac{d^2 \vect{r}_c}{dt^2}
    }
\end{equation}

Nel caso in cui il sistema sia formato da una distribuzione continua di massa $dm = \rho dV$ il suo centro di massa verrà definito dal vettore $\vect{r} = x \vect{i} + y \vect{j} + z \vect{k}$ come

\begin{equation}
    \highlight{
        \vect{r} = \frac{1}{m} \int \rho \vect{r} \ dV
    }
\end{equation}



\subsection{Quantità di moto di un sistema}

La quantità di moto di un sistema di $n$ punti materiali è definita come la somma della quantità di moto dei singoli punti materiali

\begin{equation}
    \vect{p} = {\sum}_i \vect{p}_i = {\sum}_i m_i \vect{v}_i = {\sum}_i m_i \frac{d \vect{r}_i}{dt}
\end{equation}

Per la definizione di centro di massa e dato ${\sum}_i m_i \frac{d \vect{r}_i}{dt} = m \frac{d \vect{r_c}}{dt} = m \vect{v}_c$ è possibile scrivere

\begin{equation}
    \vect{p} = m \vect{v}_c
\end{equation}

La \textcolor{accent}{prima equazione cardinale della dinamica dei sistemi} è

\begin{equation}
    \highlight{
        \vect{F} = \frac{d\vect{p}}{dt}
    }
\end{equation}



\subsection{Principio di conservazione della quantità di moto}

Se in un sistema di punti materiali in ogni istante è nulla la somma delle forze esterne (reali e apparenti) allora

\begin{equation}
    \frac{d \vect{p}}{dt} = 0
\end{equation}

e quindi $\sum \vect{p}_i = \vect{p} = \mathrm{const}$.

\begin{quote}
    La quantità di moto di un sistema isolato di punti materiali resta costante.
\end{quote}



\subsection{Teorema del momento della quantità di moto}

In un sistema formato da $n$ punti materiali, scelto un polo $O$ il \textcolor{accent}{momento risultante} rispetto a $O$ sarà

\begin{equation}
    \vect{M} = \vect{M}_1 + \vect{M}_2 + \ldots + \vect{M}_n  = \sum \vect{r}_j \times \vect{F}_j
\end{equation}

e il \textcolor{accent}{momento della quantità di moto} del sistema rispetto a $O$ sarà

\begin{equation}
    \vect{b} = \vect{b}_1 + \vect{b}_2 + \ldots + \vect{b}_n = \sum \vect{r}_j \times \vect{p}_j
\end{equation}

Preso in considerazione un solo punto materiale, vale la relazione

\begin{equation}
    \label{eq:}
    \vect{M}_j^{\mathrm{(est)}} + \vect{M}_j^{\mathrm{(int)}} = \frac{d \vect{b}_j}{dt}
\end{equation}

e nel caso in cui il punto di riferimento $O$ si muova con velocità $\vect{v}_0$, vale la relazione

\begin{equation}
    \vect{M}_j^{\mathrm{(est)}} + \vect{M}_j^{\mathrm{(int)}} = \frac{d \vect{b}_j}{dt} + \vect{v}_0 \times m_j \vect{v}_j
\end{equation}

Il momento risultante delle forze interne è nullo $\sum \vect{M}_j^{\mathrm{(int)}} = 0$ , di conseguenza, posto $\vect{M} = \sum \vect{M}_j = \sum \vect{M}_j^{\mathrm{(est)}}$, si avrà

\begin{equation}
    \highlight{
        \vect{M} = \frac{d \vect{b}}{dt}
    }
\end{equation}

e nel caso in cui $O$ sia in movimento

\begin{equation}
    \highlight{
        \vect{M} = \frac{d \vect{b}}{dt} + \vect{v}_0 \times \vect{p}
    }
\end{equation}

%

\begin{equation}
    \frac{db_a}{dt} = M_a
\end{equation}

%

\begin{equation}
    b_{aj} = m_j r_j^2 \omega_{ja} = m_j r_j^2 \dot{\varphi}_j
\end{equation}

%

\begin{equation}
    b_a = \left( \sum m_j r_j^2 \right) \omega_a = I_a \omega_a = I_a \dot{\varphi}
\end{equation}

%

\begin{equation}
    \highlight{
        I_a = \sum m_j r_j^2
    }
\end{equation}



\subsection{Teorema del lavoro nei sistemi di punti}

\begin{equation}
    L_j = \int_1^2 (\vect{F}_j + \vect{f}_j) \cdot d \vect{s}_j = \frac{1}{2} m (v_j'')^2 - \frac{1}{2} m (v_j')^2
\end{equation}

\begin{equation}
    \highlight{
        L = \sum L_j = T_2 - T_1
    }
\end{equation}

\begin{equation}
    \highlight{
        T = \frac{1}{2} \sum m_j v_j^2
    }
\end{equation}



\subsection{Teorema di K\"onig}

\begin{quote}
    L'energia cinetica di un sistema di punti è pari all'energia cinetica propria nel sistema del centro di massa più l'energia di traslazione del centro di massa
\end{quote}

\begin{equation}
    \highlight{
        T = \frac{1}{2} m v_c^2 + \frac{1}{2} \sum m_j v_{cj}^2
    }
\end{equation}



\subsection{Energia potenziale}

Dato il lavoro di forze interne ed esterne per passare dalla configurazione $1$ alla configurazione $2$

\begin{equation}
    L = \sum \int_1^2 (\vect{F}_j + \vect{f}_j) \ d\vect{s}_j
\end{equation}

e considerata una funzione per definire l'energia potenziale del sistema 

\begin{equation}
    U(x_1, y_1, z_1, \ldots, x_n, y_n, z_n)
\end{equation}

perché si passi dalla configurazione $1(x_1, y_1, z_1, \ldots, x_n, y_n, z_n)$ alla configurazione $2(x_1', y_1', z_1', \ldots, x_n', y_n', z_n')$ vale la relazione

\begin{equation}
    U(1) - U(2) = L_{12}
\end{equation}

di conseguenza, per una configurazione qualsiasi $C(x_1, y_1, z_1, \ldots, x_n, y_n, z_n)$ l'energia potenziale sarà

\begin{equation}
   U(x_1, y_1, z_1, \ldots, x_n, y_n, z_n) = - L_{1C} + U(1)
\end{equation}

Quando tutte le forze interne sono conservative, come in questo caso, il lavoro totale delle forze applicate $\vect{R}_j = \vect{F}_j + \vect{f}_j$ è un differenziale del lavoro totale

\begin{equation}
    dL = \sum (R_{jx} dx_j + R_{jy} dy_j + R_{jz} dz_j) = -dU
\end{equation}

inoltre

\begin{equation}
    \begin{dcases}
        R_{jx} = - \frac{\partial U}{\partial x_j} \\
        R_{jy} = - \frac{\partial U}{\partial y_j} \\
        R_{jz} = - \frac{\partial U}{\partial z_j}
    \end{dcases}
\end{equation}



\subsection{Conservazione dell'energia meccanica}

Nel caso in cui tutte le forze agenti sul sistema siano conservative, l'energia del sistema si conserva

\begin{equation}
    \highlight{
        T + U = E = \mathrm{const}
    }
\end{equation}



\subsection{Urto normale centrale}

Negli urti si conserva sempre la quantità di moto

\begin{equation}
    m_1 \vect{v}_{1i} + m_2 \vect{v}_{2i} = 
    m_1 \vect{v}_{1f} + m_2 \vect{v}_{2f}
\end{equation}



\subsubsection{Urto perfettamente elastico}

L'energia cinetica si conserva solo negli urti perfettamente elastici

\begin{equation}
    \frac{1}{2} m_1 \vect{v}_{1i}^2 + m_2 \vect{v}_{2i}^2 = 
    \frac{1}{2} m_1 \vect{v}_{1f}^2 + m_2 \vect{v}_{2f}^2
\end{equation}



\subsubsection{Urto anelastico}

Nel caso in cui l'urto sia \emph{perfettamente} anelastico

\begin{equation}
    m_1 \vect{v}_{1i} + m_2 \vect{v}_{2i} = 
    (m_1  + m_2) \vect{v}_{f}
\end{equation}

\begin{equation}
    (1 - \alpha) \left( \frac{1}{2} m_1 \vect{v}_{1i}^2 + m_2 \vect{v}_{2i}^2 \right) = 
    \frac{1}{2} (m_1 + m_2) \vect{v}_{f}^2
\end{equation}