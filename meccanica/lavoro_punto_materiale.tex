\section{Lavoro ed energia per il punto materiale}



\subsection{Lavoro}

Data una forza $\vect{F}$ costante applicata a un punto materiale e lo spostamento $\Delta s$ del punto materiale il lavoro è definito come

\begin{equation}
    L = \vect{F} \cdot \Delta \vect{s}
\end{equation}

In generale il lavoro elementare di una forza $\vect{F}$ per uno spostamento $d\vect{s}$ è

\begin{equation}
    \highlight{dL = \vect{F} \cdot d \vect{s}} = F_x dx + F_y dy + F_z dz
\end{equation}

e quindi il lavoro tra due punti $1$ e $2$ di una curva $C$ rappresentante lo spostamento sarà

\begin{equation}
    L = {\int_1^2}_C \vect{F} \cdot d \vect{s} = {\int_1^2}_C (F_x dx + F_y dy + F_z dz)
\end{equation}

Il lavoro, poiché $d \vect{s} = \vect{v} dt$, può essere espresso in funzione del tempo in un intervallo ($t_0$, $t$) come

\begin{equation}
    L = \int_{t_0}^t \vect{F} \cdot \vect{v} dt
\end{equation}

Il lavoro è una grandezza scalare, le dimensioni sono

\begin{equation}
    [L^2MT^{-2}]
\end{equation}



\subsection{Potenza}

La potenza media si definisce come

\begin{equation}
    \overline{W} = \frac{L}{t_2 - t_1}
\end{equation}

La potenza istantanea è definita da

\begin{equation}
    \highlight{W = \frac{dL}{dt}}
\end{equation}

Con $d \vect{s} = \vect{v} dt$

\begin{equation}
    W = \vect{F} \cdot \frac{d \vect{s}}{dt} = \vect{F} \cdot \vect{v}
\end{equation}

Le dimensioni sono

\begin{equation}
    [W] = [L^2MT^{-3}]
\end{equation}



\subsection{Energia cinetica}

\begin{equation*}
    dL = \vect{F} \cdot \Delta \vect{s} = m \frac{d \vect{v}}{dt} \cdot \vect{v} dt
\end{equation*}

\begin{equation*}
    \frac{d(v^2)}{dt} = \frac{d(\vect{v} \cdot \vect{v})}{dt} = \frac{d \vect{v}}{dt} \cdot \vect{v} + \vect{v} \cdot \frac{d \vect{v}}{dt} = 2 \frac{d \vect{v}}{dt} \vect{v}
\end{equation*}

\begin{equation*}
    dL = \frac{m}{2} \frac{d(v^2)}{dt} dt = \frac{m}{2} d (v^2) = d \left( \frac{1}{2} m v^2 \right) = dT
\end{equation*}

L'\textcolor{accent}{energia cinetica} sarà quindi definita come

\begin{equation}
    \highlight{
        T_1 = \frac{1}{2} m v^2
    }
\end{equation}

Il lavoro compiuto da una forza $\vect{F}$ su un punto materiale facendolo passare da una posizione $1$ a una posizione $2$ ne fa variare anche l'energia cinetica

\begin{equation*}
    L_{12} = \int_1^2 dL = \int_1^2 dT = T_2 - T_1 = \Delta T_{12}
\end{equation*}

quindi

\begin{equation}
    L = \Delta T
\end{equation}

Nel caso in cui il punto sia in rotazione con velocità $\omega$ e raggio $r$ l'energia cinetica si scrive

\begin{equation}
    T = \frac{1}{2} m v^2 = \frac{1}{2} (m r^2) \omega^2
\end{equation}



\subsection{Campi conservativi}

In un campo di forze conservativo il lavoro compiuto per spostare un punto materiale da una posizione $P_i$ a un'altra $P_f$ non dipende dal percorso seguito ma unicamente dalla posizione finale e da quella iniziale



\subsection{Energia potenziale}

\begin{equation}
    \highlight{
        L = - \Delta U
    }
\end{equation}

\begin{equation}
    [U] = [L^2MT^{-2}]
\end{equation}

\begin{equation}
    \vect{F} = - \left( \frac{\partial U}{\partial x} \vect{i} +
                        \frac{\partial U}{\partial y} \vect{j} +
                        \frac{\partial U}{\partial z} \vect{k} \right) = - grad U = - \nabla U
\end{equation}




\subsubsection{Campo di forze gravitazionali}

Nel caso di forze $\vect{F} = m \vect{g}$

\begin{align*}
    U(x,y,z) &= U(y) = - \int_{y_0}^y mg \ dy + U(y_0)  \\
    &= -mgy + mgy_0 + U(y_0) =                          \\
    &= -mgy + \mathrm{const}
\end{align*}

Ponendo $U = 0$ per $y = 0$

\begin{equation}
    \highlight{
        U(y) = - mgy
    }
\end{equation}



\subsubsection{Campo di forze centrali}

Nel caso di forze gravitazionali e elettrostatiche $\vect{F} (r) = \pm k/r^2 \ \vect{r} / r$

\begin{align*}
    U(r) &= - \int_{r_0}^r \left( \pm \frac{k}{r^2} dr \right) + \mathrm{const} = \\
    &= \pm \left( \frac{1}{r} - \frac{1}{r_0} \right) + U(r_0) =
    &= \pm \frac{k}{r} + \mathrm{const}
\end{align*}

Ponendo $U(r) = 0$ per $r = \infty$

\begin{equation}
    \highlight{
        U(r) = \pm \frac{k}{r}
    }
\end{equation}



\subsubsection{Campo di forze elastiche}

In questo caso $\vect{F} (x) = - k x \vect{i}$ e l'energia potenziale elastica sarà pari a

\begin{equation}
    U(x) = - \int_{x_0}^x (- k x) dx + U(x_0) = \frac{1}{2} k x^2 + \mathrm{const}
\end{equation}

con la costante nulla nel caso si scelga di porre $U(x) = 0$ per $x = 0$.



\subsection{Conservazione dell'energia meccanica con forze conservative}

Nel caso di un punto materiale soggetto a forze conservative la somma dell'energia cinetica e dell'energia potenziale rimane invariata durante il moto.

\begin{equation}
    \highlight{E = \mathrm{const} = T_1 + U_1 = T_2 + U_2}
\end{equation}



\subsection{Variazione dell'energia meccanica in presenza di forze non conservative}

Nel caso generale per cui siano presenti forze conservative e non conservative il lavoro che porta un punto materiale dalla posizione $1$ alla posizione $2$ può essere espresso come

\begin{equation}
    \sum L = \sum L_{nc} + \sum L_c = T_2 - T_1
\end{equation}

Poiché il lavoro delle forze conservative può essere espresso attraverso la variazione di energia potenziale $(U_2 - U_1)$, il lavoro delle forze non conservative si può definire come

\begin{equation}
    \sum L_{nc} = (T_2 + U_2) - (T_1 + U_1) = E_2 - E_1
\end{equation}